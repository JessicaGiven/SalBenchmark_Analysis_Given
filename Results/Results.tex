\documentclass[journal]{IEEEtran}

\usepackage{cite}
\usepackage[pdftex]{graphicx}
\DeclareGraphicsExtensions{.pdf,.jpg,.png}
\usepackage[pdfborder=false,colorlinks,bookmarksnumbered,bookmarksopen,linkcolor=black,citecolor=black,urlcolor=black]{hyperref}
\usepackage[tight,footnotesize]{subfigure}
\usepackage{rotating}
\usepackage{graphicx}
\usepackage{amsmath,amssymb} % define this before the line numbering.
\usepackage{epsfig}
\usepackage{array}
\usepackage{multirow}
\usepackage{colortbl}
\usepackage{amsfonts}
\newcommand{\tickYes}{\bullet}
\newcommand{\cmark}{\ding{51}}
\usepackage{pifont}
\newcommand{\tickNo}{\hspace{1pt}\ding{55}}
\definecolor{gray1}{rgb}{.8,.8,.8}
\usepackage{xspace}
\usepackage{footnote}
\usepackage{etoolbox}
\usepackage{overpic}
\usepackage{multirow}
\usepackage{multicol}

\def\ie{\emph{i.e.}}
\def\eg{\emph{e.g.}}
\def\etal{{\em et al.}}

\newcommand{\cmm}[1]{{\textcolor{blue}{#1}}}
\newcommand{\ali}[1]{{\textcolor{red}{[ALI:] #1}}}
\newcommand{\hzj}[1]{{\textcolor{magenta}{[HZ]: #1}}}
\newcommand{\first}[1]{{\textcolor{red}{#1}}}
\newcommand{\second}[1]{{\textcolor{green}{#1}}}
\newcommand{\third}[1]{{\textcolor{blue}{#1}}}
\newcommand{\figref}[1]{Fig. \ref{#1}}
\newcommand{\tabref}[1]{Fig. \ref{#1}}
\newcommand{\equref}[1]{(\ref{#1})}
\newcommand{\secref}[1]{Sec. \ref{#1}}
\newcommand{\algref}[1]{Alg. \ref{#1}}
\renewcommand{\arraystretch}{1.1}
\renewcommand{\tabcolsep}{.5mm}
\newcommand{\datasets}{{\scriptsize PASCAL-S}&{\scriptsize THUR15K} & {\scriptsize JuddDB} &
{\scriptsize DUT-OMRON}&{\scriptsize MSRA10K} &
{\scriptsize ECSSD}}

\newcommand{\Rows}[1]{\multirow{3}{*}{#1}}
\newcommand{\Cols}[1]{\multicolumn{3}{c|}{#1}}
\newcommand{\tabTitle}{}
\newcommand{\tabincell}[2]{\begin{tabular}{@{}#1@{}}#2\end{tabular}}
\newcommand{\AddImg}[1]{}
\newcommand{\AddImgs}[2]{}
\newcommand{\AddImgsU}[2]{}
\newcommand{\myPara}[1]{\vspace{.05in}\noindent\textbf{#1.}}
\newcommand{\sArt}{state-of-the-art}



\ifCLASSINFOpdf
\else
\fi

\begin{document}



\begin{figure*}[t]
\renewcommand{\tabTitle}{ \textbf{Model}&\scriptsize {Dataset1}&\scriptsize {Dataset2}}
\begin{minipage}[t]{0.48\linewidth}
    \centering \small
    \renewcommand{\arraystretch}{1.1}
    \renewcommand{\tabcolsep}{1.2mm}
    \begin{tabular}{|l||c|c|} \hline
	\tabTitle \\	\textbf{FT}   & 0.682 & 0.801 \\
	\textbf{GC}   & 0.820 & 0.933 \\
	\textbf{HC}   & 0.774 & 0.880 \\
	\textbf{RC}   & 0.783 & \first{0.980} \\
	\textbf{GMR}  & \first{0.930} & \second{0.976} \\
	\textbf{GU}   & 0.804 & 0.929 \\
	\textbf{MSS}  & 0.835 & 0.910 \\
	\textbf{PCA}  & \second{0.916} & \third{0.965} \\
	\textbf{SWD}  & \third{0.866} & 0.939 \\
\hline
\end{tabular}

    \caption{AUC: area under ROC curve (Higher is better. The top three models are highlighted in red, green and blue). }\label{fig:AUC}
\end{minipage}
\hspace{0.9cm}
\begin{minipage}[t]{0.48\linewidth}
    \centering \small
    %\fontsize{8.5}{1em}\selectfont
    \renewcommand{\arraystretch}{1.1}
    \renewcommand{\tabcolsep}{1.2mm}
    \begin{tabular}{|l||c|c|} \hline
	\tabTitle \\	\textbf{FT}   & 0.269 & 0.250 \\
	\textbf{GC}   & \second{0.159} & \third{0.114} \\
	\textbf{HC}   & 0.278 & 0.214 \\
	\textbf{RC}   & 0.233 & \first{0.092} \\
	\textbf{GMR}  & \first{0.133} & \second{0.107} \\
	\textbf{GU}   & \third{0.183} & 0.122 \\
	\textbf{MSS}  & 0.207 & 0.208 \\
	\textbf{PCA}  & 0.208 & 0.176 \\
	\textbf{SWD}  & 0.294 & 0.260 \\
\hline
\end{tabular}

    \caption{MAE: Mean Absolute Error (Smaller is better.  The top three models are highlighted in red, green and blue).}\label{fig:MAE}
\end{minipage}
\end{figure*}



\newcommand{\fMeasureTypes}{Fixed & AdpT & SCut}
\renewcommand{\tabTitle}{{\textbf{Model}}&\Cols{\textbf{Dataset1}}&\Cols{\textbf{Dataset1}}}


\begin{figure*}[t]
\begin{minipage}[t]{\linewidth}
    \centering
    \small
    \renewcommand{\arraystretch}{1.1}
    \renewcommand{\tabcolsep}{0.505mm}
    \begin{tabular}{|l||ccc|ccc|} \hline
	\tabTitle \\	\textbf{FT}   & 0.507 & 0.464 & 0.558 & 0.673 & 0.595 & 0.655 \\
	\textbf{GC}   & \second{0.698} & \third{0.683} & \third{0.759} & 0.833 & \third{0.843} & 0.818 \\
	\textbf{HC}   & 0.540 & 0.575 & 0.625 & 0.734 & 0.654 & 0.827 \\
	\textbf{RC}   & 0.552 & 0.497 & 0.553 & \first{0.911} & \first{0.866} & \second{0.947} \\
	\textbf{GMR}  & \first{0.819} & \first{0.709} & \first{0.817} & \second{0.903} & 0.841 & \first{0.948} \\
	\textbf{GU}   & 0.645 & 0.638 & 0.685 & 0.834 & 0.815 & 0.864 \\
	\textbf{MSS}  & 0.599 & 0.620 & 0.628 & 0.773 & 0.760 & 0.800 \\
	\textbf{PCA}  & \third{0.698} & \second{0.684} & \second{0.773} & \third{0.844} & \second{0.862} & 0.882 \\
	\textbf{SWD}  & 0.608 & 0.549 & 0.694 & 0.765 & 0.750 & \third{0.918} \\
\hline
\end{tabular}

    \caption{$F_\beta$ statistics on each dataset,
        using varying fixed thresholds,
        adaptive threshold, and SaliencyCut (Higher is better. The top three models are highlighted in red, green and blue).
    }\label{fig:FMeasure}
\end{minipage}
\end{figure*}



\begin{figure*}[!htbp]
    \renewcommand{\arraystretch}{1}
    \renewcommand{\tabcolsep}{1.272mm}
    \footnotesize
    \begin{tabular}{|l||c|c|c|c|c|c|c|c|c|} \hline
	Method &   FT&   GC&   HC&   RC&  GMR&   GU&  MSS&  PCA&  SWD\\\hline
	Max   & 9 & \third{3} & 8 & 4 & \first{1} & 5 & 6 & \second{2} & 7 \\
	AUC   & 9 & 5 & 8 & 4 & \first{1} & 7 & 6 & \second{2} & \third{3} \\
	MAE   & 7 & \second{2} & 8 & \third{3} & \first{1} & 4 & 5 & 6 & 9 \\
	AdpT  & 9 & \third{3} & 7 & 4 & \second{2} & 5 & 6 & \first{1} & 8 \\
	SCut  & 9 & 4 & 7 & 6 & \first{1} & 5 & 8 & \second{2} & \third{3} \\
\hline
\end{tabular}
 \caption{Summary rankings of models under different evaluation metrics.
        Fixation prediction models are shown in bold face. The top three models under each evaluation metric are highlighted in red, green and blue.
%        \cmm{Personally I prefer to remove the second and third row as they are similar to first row,
%       and is more influenced by the parameter we choose for these segmentation method.}
    }\label{tab:summaryRank}
\end{figure*}



\end{document}

